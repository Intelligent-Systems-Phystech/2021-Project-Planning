\documentclass[12pt, twoside]{article}
\usepackage{jmlda}
\newcommand{\hdir}{.}

\begin{document}

\title
    [Экспериментальное сравнение задач и моделей планирования биохимического производства.] % краткое название; не нужно, если полное название влезает в~колонтитул
    {Экспериментальное сравнение задач и моделей планирования биохимического производства.}
\author
    [В.\,В.~Пырэу, С.\,А.~Тренин] % список авторов (не более трех) для колонтитула; не нужен, если основной список влезает в колонтитул
    {В.\,В.~Пырэу, С.\,А.~Тренин} % основной список авторов, выводимый в оглавление
\email
    {kondratiukvitalik@gmail.com; s.trenin@gmail.com}
\abstract
    {Целью данной научной работы является комплексное исследование задачи оперативного планирования производства для биохимической промышленности. Исследуются различные постановки задачи составления оптимальных расписаний, учитывающие различные ограничения, приходящие из практики: особенности хранения промежуточных веществ, требования к работе производственных узлов и особенности подготовки станков, такие как наладка и очистка между запусками. Основной класс рассматриваемых математических моделей - смешанное целочисленное линейное программирование, что означает, что рассматриваемые задачи являются $\NP$-трудными. Сложность заключается в том, что схожие задачи могут решаться одинаковыми методами с разной степенью эффективности, что негативно сказывается на процессе внедрения моделей в практику. Для решения этого вопроса проводится экспериментальный запуск моделей, разработанных для одной предметной области, в задачах из другой, интерпретация полученных результатов и предложение эвристик для ускорения алгоритмов.
    
    
\bigskip
\noindent
\textbf{Ключевые слова}: \emph {planning; sheduling; MILP}
}

%данные поля заполняются редакцией журнала
\doi{}
\receivedRus{}
\receivedEng{}

\maketitle
\linenumbers

\section{Введение}
На сегодняшний день наблюдается высокая конкуренция в разных областях биохимического производства, а так же усложнение производственных процессов, увеличение числа этапов, количества оборудования и объемов продукции. Это неизбежно влечёт появление естественных требований к алгоритмам планирования: они должны быть масштабируемыми, работать за разумное время, находить качественные приближения к оптимальному расписанию и быть гибкими к изменению начальных условий. Основной объект исследования --- это различные варианты постановки задачи создания расписания как задачи оптимизации, методы её решения и эвристики, учитывающие индивидуальные особенности задач. На данный момент широкое распространение получили модели смешнанного целочисленного линейного программирования(ЦЛП), так как соответствующие задачи хорошо изучены и существуют алгоритмы для их решения. Однако временные затраты и степень оптимальности найденных решений сильно зависят от количества переменных и ограничений в модели, что делает процесс моделирования значимым для создания плана. 

Большинство статей в данной области посвящены конкретным постановкам задач, приходящим из практики и созданию конкретных моделей для их решения. При этом задачи сходны друг другу, хоть и принадлежат разным предметным областям: фармацевтической, пищевой, химической и др. Это, в свою очередь, позволяет применять идеи, высказанные для решения одной задачи к решению другой. Применение имеющихся моделей и эвристик к задачам, для которых они не были разработаны изначально позволит перенять имеющийся опыт, а так же провести тонкую настройку модели под конкретную постановку, что должно привести к улучшению качества. 

Некоторые авторы предлагают пути упрощения модели с целью ускорить процесс получения результата без значительной потери его качества. Примером подобной эвристики является двухступенчатая схема, представленная в \cite{lpheuristic}. В работе проводится анализ других способов упрощения модели и сравнительная оценка результатов.

\section{Название раздела}
Данный документ демонстрирует оформление статьи,
подаваемой в электронную систему подачи статей \url{http://jmlda.org/papers} для публикации в журнале <<Машинное обучение и анализ данных>>.
Более подробные инструкции по~стилевому файлу \texttt{jmlda.sty} и~использованию издательской системы \LaTeXe\
находятся в~документе \texttt{authors-guide.pdf}.
Работу над статьёй удобно начинать с~правки \TeX-файла данного документа.

Обращаем внимание, что данный документ должен быть сохранен в кодировке~\verb'UTF-8 without BOM'.
Для смены кодировки рекомендуется пользоваться текстовыми редакторами \verb'Sublime Text' или \verb'Notepad++'.

\paragraph{Название параграфа}
Разделы и~параграфы, за исключением списков литературы, нумеруются.

\section{Заключение}
Желательно, чтобы этот раздел был, причём он не~должен дословно повторять аннотацию.
Обычно здесь отмечают, каких результатов удалось добиться, какие проблемы остались открытыми.

%%%% если имеется doi цитируемого источника, необходимо его указать, см. пример в \bibitem{article}
%%%% DOI публикации, зарегистрированной в системе Crossref, можно получить по адресу http://www.crossref.org/guestquery/
\begin{thebibliography}{99}

\bibitem{lpheuristic}
    \BibAuthor{F. Blomer, H.-O. Gunther}
    LP-based heuristics for scheduling chemical batch processes~//
    \BibJournal{International Journal of Production Research}, 38:5, 1029-1051
	\BibDoi{10.1080/002075400189004}.

%\bibitem{webArticle}
%	\BibAuthor{Blaga~P.\,A.}
%	Commutative Diagrams with XY-pic II. Frames and Matrices~//
%	\BibJournal{PracTEX J.}, 2007. Vol.\,4.
%	URL: \BibUrl{https://tug.org/pracjourn/2007-1/blaga/blaga.pdf}.
%
%\bibitem{webResource}
%	XYpic.
%	URL: \BibUrl{http://akagi.ms.u-tokyo.ac.jp/input9.pdf}.
%	
%\bibitem{inproceedingsRus}
%	\BibAuthor{Усманов~Т.\,С., Гусманов~А.\,А., Муллагалин~И.\,З., Мухаметшина~Р.5\,Ю., Червякова~А.\,Н., Свешников~А.\,В.}
%	Особенности проектирования разработки месторождений с применением гидроразрыва пласта~//
%	\BibJournal{Труды 6-го Междунар. симп. <<Новые ресурсосберегающие технологии недропользования и повышения нефтегазоотдачи>>}.~---
%	М.:~Издательство, 2007. С.~267--272.

%\bibitem{inproceedingsEng}
 %   \BibAuthor{Author~N.}
  %  Paper title~//
   % \BibJournal{10th Conference (International) on Any Science Proceedings}.~---
    %Place of publication: Publisher, 2009. P.~111--122.

%\bibitem{techreport}
%	\BibAuthor{Lambert~P.}
 % 	\BibTitle{The title of the work}.
  %	Place of publication:~The institution that published, 1993.  Report~2.
 	
\end{thebibliography}

%%%% если имеется doi цитируемого источника, необходимо его указать, см. пример в \bibitem{article}
%%%% DOI публикации, зарегистрированной в системе Crossref, можно получить по адресу http://www.crossref.org/guestquery/.

\end{document}
