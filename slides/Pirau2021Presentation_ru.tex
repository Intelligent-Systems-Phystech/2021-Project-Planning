\documentclass{beamer}
\beamertemplatenavigationsymbolsempty
\usecolortheme{beaver}
\setbeamertemplate{blocks}[rounded=true, shadow=true]
\setbeamertemplate{footline}[page number]
%
\usepackage[utf8]{inputenc}
\usepackage[english,russian]{babel}
\usepackage{amssymb,amsfonts,amsmath,mathtext}
\usepackage{subfig}
\usepackage[all]{xy} % xy package for diagrams
\usepackage{array}
\usepackage{multicol}% many columns in slide
\usepackage{hyperref}% urls
\usepackage{hhline}%tables
% Your figures are here:
\graphicspath{ {fig/} {../fig/} }

%----------------------------------------------------------------------------------------------------------
\title[\hbox to 56mm{Задачи планирования}]{	Экспериментальное сравнение \\ задач оперативного планирования биохимического производства.}
\author[В.\,В. Пырэу]{Виталий Вячеславович Пырэу}
\institute{Московский физико-технический институт}
\date{\footnotesize
\par\smallskip\emph{Курс:} Автоматизация научных исследований\par (практика, В.\,В.~Стрижов)/Группа Б05-821
\par\smallskip\emph{Эксперт:} С.\,А.~Тренин
\par\bigskip\small 2021}
%----------------------------------------------------------------------------------------------------------
\begin{document}
%----------------------------------------------------------------------------------------------------------
\begin{frame}
\thispagestyle{empty}
\maketitle
\end{frame}
\begin{frame}{Моделирование времени, ограничений и требований к расписанию}

Как кодировать последовательность запусков процессов?

\begin{enumerate}
	\item Дискретный индикатор времени $s_{i, t} = 1$ если задача $i$ началась в момент $t$.
	\item Прореживание сетки + сдвиг влево (только $t$ кратные $t_0$)
	\item Переменные длины временных промежутков.
	\item Индикатор порядка задач: $b_{i, j} = 1$, если задача $i$ начинается раньше задачи $j$.
\end{enumerate}

\includegraphics[width=1.0\textwidth]{Plan}

\end{frame}
%----------------------------------------------------------------------------------------------------------
\end{document} 